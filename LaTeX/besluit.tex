\chapter{Besluit}


Het uitgebreid onderzoek in hoofdstuk \ref{privacyklassiek} gaf een goed inzicht in de soorten privacygevoelige data die klassieke aanbevelingssystemen bijhouden en welke privacyrisico's ze lopen. Dit inzicht hielp om in te schatten hoe de besproken bestaande privacyvriendelijke oplossingen scoren op dit gebied. Naast privacy zijn natuurlijk ook de performantie en nauwkeurigheid van een werkwijze belangrijk. Er werd ook een onderscheid gemaakt tussen de performantie op de client en de performantie op de eventuele servers. Aangezien de gekozen werkwijze op een mobiel toestel hoeft te werken ligt er een hogere prioriteit op de performantie op de client. Rekening houdend met al deze criteria werd er gekozen om de werkwijze te baseren op het artikel "Privacy-preserving recommender systems in dynamic environments \cite{ZErkinDyn}" van Z. Erkin, T. Veugen en R.L. Lagendijk. De protocollen waarop in deze paper de gelijkenissen berekend worden en hoe de waarden privacyvriendelijk opgeteld, vermenigvuldigd en naar de gebruiker gebracht werden aan de hand van cryptosystemen, werden overgenomen uit dit artikel. Er waren wel nog een aantal praktische uitdagingen rekening houdende met het mobiel platform. Het artikel zelf berekent de aanbevelingen echter op een erg primitieve manier, door het (niet-gewogen) gemiddelde te nemen van de beoordelingen van gelijkaardige gebruikers. De gebruikte cryptografische methode en het mobiel platform staan echter ook een andere manier toe, die nauwkeurigere aanbevelingen toelaat zonder daarvoor veel extra werk te doen. Deze werkt ongeveer even nauwkeurig als een privacy-onvriendelijke methode. Het hele systeem is even privacyvriendelijk als het origineel en biedt dus de beste garantie op privacy van alle besproken methodes. De belasting van het mobiele toestel is ook beperkt. Enkel de performantie is nog een werkpunt



Op basis 
diepgaande analyse over de bestaande methodes met een 