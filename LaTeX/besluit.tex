\chapter{Besluit}


Het uitgebreid onderzoek in hoofdstuk \ref{privacyklassiek} gaf een goed inzicht in de soorten privacygevoelige data die klassieke aanbevelingssystemen bijhouden en welke privacyrisico's ze lopen. Dit inzicht hielp om in te schatten hoe de besproken bestaande privacyvriendelijke oplossingen scoren op dit gebied. Naast privacy zijn natuurlijk ook de performantie en nauwkeurigheid van een werkwijze belangrijk. Er werd een onderscheid gemaakt tussen de performantie op de client en de performantie op de eventuele servers. Aangezien de gekozen werkwijze op een mobiel toestel hoeft te werken ligt er een hogere prioriteit op de performantie op de client. Rekening houdend met al deze criteria werd er gekozen om de methode te baseren op het artikel "Privacy-preserving recommender systems in dynamic environments \cite{ZErkinDyn}" van Z. Erkin, T. Veugen en R.L. Lagendijk. De protocollen waarmee in deze paper de gelijkenissen berekend worden en de waarden privacyvriendelijk opgeteld, vermenigvuldigd en naar de gebruiker gebracht, werden overgenomen uit dit artikel. Er waren wel nog een aantal praktische uitdagingen rekening houdende met het mobiel platform. Het artikel zelf berekent de aanbevelingen echter op een erg primitieve manier, door het (niet-gewogen) gemiddelde te nemen van de beoordelingen van gelijkaardige gebruikers. De gebruikte cryptografische methode en het mobiel platform staan echter ook een andere manier toe die een waarde berekent gewogen ten opzichte van het gemiddelde van de gebruikers. Deze methode, gebaseerd op een klassieke formule, laat nauwkeurigere aanbevelingen toe zonder daarvoor veel extra werk te doen. Ze werkt ongeveer even nauwkeurig als een privacy-onvriendelijke methode. Het hele aanbevelingssysteem is even privacyvriendelijk als het origineel en biedt dus de beste garantie op privacy van alle besproken methodes. De belasting van een mobiele toestel is beperkt. De performantie van het algoritme op de servers daarentegen is nog een werkpunt. De gebruikte oplossing heeft ook een derde partij nodig die de rol van controleserver kan vervullen in het protocol. Deze zou vervuld kunnen worden door de overheid of een privaat beveiligingsbedrijf. De derde partij zou ook de applicatie moeten nagaan en voorkomen dat de applicatie zelf op een andere, niet-ge\"encrypteerde manier, de gegevens zou doorgeven. Er kan besloten worden dat een privacyvriendelijk algoritme zoals hier besproken, waarbij de serverwerking nog wat geoptimaliseerd wordt, het best beantwoordt aan de noden van een aanbevelingssysteem voor mobiele toestellen.