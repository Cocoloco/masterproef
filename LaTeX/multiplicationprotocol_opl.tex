\subsection{Multiplicatieprotocol}
\label{multiplication} 
Hier vermenigvuldigt de recommenderserver de ratings en bitvalues van de gebruiker met hun resultaatbits uit \ref{selection} die aangeven of hun beoordeling worden opgenomen in het protocol. De vermenigvuldiging van twee ge\"encrypteerde waarden zit niet standaard in het Pailliercryptosysteem. Dit wordt opgelost door dit eenvoudig subprotocol \cite{erkin:generating}. De aanbevelingsserver bepaalt per vermenigvuldiging eerst twee random getallen $r_1$ en $r_2$ die hij aftrekt van de respectievelijke bits en zo $[sim_x - r_1]_2$ en $[rating_y - r_2]_2$ bepaalt. Hij stuurt deze naar de controleserver die deze decrypteert, vermenigvuldigt, weer encrypteert met zijn publieke sleutel en terugstuurt. Zo kan de recommenderserver het resultaat van de vermenigvuldiging berekenen door:

\begin{equation}
[sim_x.rating_y]_2=[(sim_x-r_1).(rating_y-r_2) + sim_x.r_2 + rating_y.r_1 - r_1.r_2]_2
\end{equation}
\begin{equation}
[sim_x.rating_y]_2= [(sim_x-r_1).(rating_y-r_2)]_2 . [sim_x]_2^{r_2} . [rating_y]_2^{r_1}.[-r_1.r_2]_2
\end{equation}
Deze berekeningen zijn allemaal mogelijk in het Pailiercryptosysteem.