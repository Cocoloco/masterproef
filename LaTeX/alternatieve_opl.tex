\subsection{Alternatieve oplossing}
De werkwijze beschreven in "Privacy-preserving recommender systems in dynamic environments \cite{ZErkinDyn}" berekent heel eenvoudig een voorspelling van het aantal sterren door het gemiddelde te nemen van het aantal sterren van gelijkaardige gebruikers, zie formule \ref{stdrecom}. De cryptografische werkwijze laat echter ruimte om met een wijziging de aanbevelingen op een betere wijze te bepalen. Dit kan via een kleine aanpassing aan een klassieke formule uit het user-user collaborative filtering.

\begin{equation}\label{stdrecom}p_{U_1,i} = \bar{r}_{U_1} + \frac{\sum_{j=2}^{n}(r_{(U_j,i)} - \bar{r}_{U_j}).s_{(U_1,U_j)}}{\sum_{j=2}^{n} s_{(U_1,U_j)}}
\end{equation}

Zo worden de aanbevelingen berekend ten opzichte van de gemiddeldes van de gebruiker zelf $\bar{r}_{U_1}$ en van de anderen $\bar{r}_{U_j}$. De $prediction$ wordt nu bepaald door het gemiddelde van de gebruiker zelf samen te tellen met het gemiddelde aantal sterren dat andere gelijkaardige gebruikers beoordeelden boven hun gemiddelde. Indien het resultaat onder 1 of boven 5 sterren gaat, worden de grenswaarden in de plaats gebruikt.
We passen dit eenvoudig in in de gebruikte cryptografische methode. De clientapplicatie stuurt enkel het ge\"encrypteerde verschil van zijn beoordeling en zijn gemiddelde op. Aangezien het Pailliercryptosysteem enkel met positieve gehele waarden werkt, wordt dit verschil wel eerst omgezet. De rest van het proces blijft gelijk, tot de gebruikerapplicatie de som van alle waarden terugkrijgt. De applicatie zet deze som om in de som van de oorspronkelijke verschillen. Deze som wordt gedeeld door het aantal gebruikte beoordelingen en samengeteld met het gemiddelde van de gebruiker zelf. Dit zou nauwkeurigere aanbevelingen moeten geven, wat zal blijken in de resultaten.