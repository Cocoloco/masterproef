
\subsection{Conclusie onderzoek}

Nu er van de belangrijkste soorten oplossingen zijn doorlicht, kan er besloten worden in welke richting we verderwerken. Een eerste optie met behulp van randomisatie leek op het eerste zicht erg interessant omdat deze heel weinig berekeningen aan de clientkant vraagt, wat natuurlijk een pluspunt is bij mobiele toestellen. Helaas werkt dit systeem niet goed bij lage gebruikersaantallen en kan het informatie lekken aan de service provider afhankelijk van de gekozen verdeling. Daar heeft de verstoring met behulp van aggregatie geen last van. Deze biedt qua privacy wel deniability of preferences aan de user maar geeft de provider wel toegang tot originele beoordelingen van de gebruiker. Als Narayan en Schmatikov in \cite{Narayanan2008} anonieme profielen kunnen linken met users met maar een beperkte kennis over deze gebruiker, lijkt het maar een kleine stap om uit het aangegeven verstoorde profiel de juiste ratings bij een gebruiker te plaatsen. Zeker indien er maar wordt geaggregeerd met een beperkt aantal andere gebruikersprofielen zoals in \cite{LCA-CONF-2009-014} beschreven. Hoewel de oplossing op basis van een sociaal netwerk erg privacyvriendelijk is, is ze niet echt flexibel. Ze vereist dat andere gebruikers online zijn en ook actief meewerken aan het protocol. Door het \emph{peer-to-peer} character van deze werkwijze zijn de clients verplicht om alle berekeningen zelf te doen. Het is duidelijk dat deze eigenschappen in een mobiele setting niet gewenst zijn.
Bij de oplossing met anonimisatie heeft de clientapplicatie daarentegen weinig werk. Het stuurt de beoordelingen rechtstreeks naar de filterentiteit, die moet vertrouwd worden. De anonimisatie die gebeurt in het user-user collaborative filtering systeem of de kennis-gebaseerde versie voldoet niet om volledig privacyvriendelijk te zijn zoals aangegeven in \ref{onterecht_vertrouwen}. De meest privacyvriendelijke oplossing lijkt de methode met behulp van cryptografische protocollen en een server. Het systeem werkt er volledig privacyvriendelijk in een statische setting. Uit verder onderzoek blijkt dat dezelfde auteur ook een paper uitbracht in november 2013 \cite{ZErkinDyn} die een erg gelijkaardige oplossing aanbiedt die ook privacyvriendelijk is in een dynamische setting. Deze methode heeft als extra voordeel dat de gebruikersapplicatie relatief weinig berekeningen hoeft te doen. Het enige minpunt is dat de berekeningen aan de serverkant erg zwaar zijn door de cryptografische protocollen. We besluiten voor deze werkwijze te kiezen en deze verder te optimaliseren in een mobiele setting in hoofdstuk \ref{privacy_opl}.