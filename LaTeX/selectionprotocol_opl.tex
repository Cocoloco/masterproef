\subsection{Selectie maken van gebruikers}
\label{selection}
Na het meerdere keren aanroepen van het aanbevelingssysteem, zou het systeem of een gebruiker kunnen afleiden de data van welke gebruikers welke invloed heeft op de aanbevelingen. Dit heet een new group attack en is eerder al vernoemd in \ref{cryptoprotocollen}. Om een new group attack tegen te gaan wordt er door de controleserver een selectie van gebruikers gemaakt die kunnen deelnemen aan het protocol. De recommender kent deze selectie dus niet. De recommender zou daarvoor de gelijkenisbits naar de controleserver moeten sturen. De bits zijn nog ge\"encrypteerd met de publieke sleutel van de controleserver, die deze bits niet mag kennen. Om dit op te lossen worden er randomgetallen $[\rho_i]_1$ bij de ge\"encrypteerde simbits geteld. Het randomgetal wordt zelf ook meegestuurd, ge\"encrypteerd met de publieke sleutel van de recommenderserver. Zo kan de controleserver $[sim_i+\rho_i]_2$ decrypteren met zijn eigen private sleutel. Deze som encrypteert hij dan weer met de publieke sleutel van de recommenderserver. Op die manier kan hij de randomwaarde hiervan aftrekken met het Pailliersysteem $[sim_i+\rho_i-\rho_i]_1$ en de originele gelijkenisbits verkrijgen. De controleserver maakt nu een tabel aan met bits met evenveel 1-waarden als gebruikers die hij wil betrekken in het protocol. In \cite{ZErkinDyn} wordt een waarde $n/2$ voorgesteld voor voldoende privacy. De server vermenigvuldigt deze bits met de gelijkenisbits en stuurt het resultaat terug naar de aanbevelingsserver. Hierbij wordt nogmaals gewisseld van cryptosysteem zoals bij het heensturen van de bits.